\documentclass[a4paper,10pt]{article}
\input{/Users/WannaGetHigh/workspace/latex/macros.tex}

\title{Semaine 2 : RVI}
\author{Fran�ois \bsc{Lepan}}

\begin{document}
\maketitle

\section{Balayage des technologies}

\begin{paragraph}{1) Ecran}~

Salle de RV

	~ $\rightarrow$ grand �cran + projecteur
\end{paragraph}


\begin{paragraph}{2) Capteurs de mouvement et position}~

Capteur de position :

	~ $\rightarrow$ 6 degr� de libert� (3 pour la position et 3 pour l'orientation)
	
Positions de nombreux degr� de libert� du corps en particulier sur la main.

\end{paragraph}

\begin{paragraph}{3) Retour d'effort}~

bras � retour d'effort

data flavor

\end{paragraph}

\begin{paragraph}{Casques}~

\begin{itemize}
\item 2 �crans
\item Tracking pr�cis de la t�te
\end{itemize}

\end{paragraph}

\begin{paragraph}{cam�ra (R�alit� augment�)}~

Superposition r�el / virtuel

2 Solutions: 

	~ $\rightarrow$ superposer les modules directement dans le r�el.
	
	~ $\rightarrow$ On film la sc�ne -> on fait des calages (c'est � dire on incline la vue dans le mod�le).
	
Bcp de reconnaissance de forme et de traitement d'image
\end{paragraph}

\begin{paragraph}{d�placements}
V�hicules

M�taphoriques (pad de jeu, p�riph�riques explicites)
\end{paragraph}

\begin{paragraph}{Principales technologies - capteurs �lectromagn�tiques}

Wiimote et acc�l�rom�tre

Cam�ra infrarouge

	~ $\rightarrow$ avec marqueurs (motion capture)
	
	~ $\rightarrow$ sans marqueurs - softKinetic / kinect (cam�ra + radar)
	
Effet de mode : leap motion	

\end{paragraph}




\end{document}